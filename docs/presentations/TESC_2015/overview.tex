% \frame { \frametitle{Hirsch-Fye Quantum Monte Carlo Method}
% The Hirsch-Fye QMC (HF-QMC) was the  method of choice used to construct QIP 
% solvers before the Continuous-time QMC (CT-QMC).
% 
% HF-QMC writes the imaginary-time functional integral by discretizing
% the interval into $M$ equally spaced “time-slices” and then on each time-slice 
% applies a discrete Hubbard-Stratonovich transformation, as shown with the single 
% orbital Anderson impurity model,
% 
% \begin{align}
% e^{-\Delta \tau U\left(n_\uparrow 
% n_\downarrow-\frac{n_\uparrow+n_\downarrow}{2}\right)}&=\frac{1}{2}\sum_{
% s_i=\pm 
% 1}e^{\lambda s_i\left(n_\uparrow-n_\downarrow\right)},\\
% \lambda&=\text{arcosh}\left[\exp\left(\frac{1}{2}\Delta\tau U\right)\right].
% \end{align}
% }
% 
% \frame { \frametitle{CT-QMC vs HF-QMC}
% CT-QMC has several advantages over HF-QMC,
% 
% \begin{itemize}
% \item HF-QMC requires an equally spaced time discretization which may 
%       cause troubles in time step extrapolation while CT-QMC doesn't have such 
% issues by design.
% \item At large interactions and low temperatures, equilibration may become an 
% issue for HF-QMC while it could be handled in CT-QMC.
% \item For system beyong single orbital, it is prohibitively difficult to handle 
% with HF-QMC 
%       while CT-QMC could be generalized to handle multiple orbital cases.
% \end{itemize}
% }

\frame { \frametitle{Basic Idea of CT-QMC}
In CT-QMC, the Hamiltonian $H=H_a+H_b$ is split into two parts.
The partition function $Z=\Tr[e^{-\beta H}]$ is written in the interaction representation 
with respect to $H_a$ and expands in powers of $H_b$,
\begin{align}
Z=&\Tr\ {T_\tau} e^{-\beta H_a} \text{exp} \left[-\int_0^\beta d\tau H_b(\tau)\right] \nonumber \\
=&\sum_k (-1)^k\int_0^\beta d\tau_1\ldots\int_{\tau_{k-1}}^\beta d\tau_k \nonumber \\
&\times \text{Tr}\big[e^{-\beta H_a}H_b(\tau_k)H_b(\tau_{k-1})\ldots H_b(\tau_1)\big]
\end{align}
The impurity Green's function ($0 < \tau < \beta$) or the ``solution'' of the impurity model is then given by
\begin{equation}
 G(\tau) = \frac{1}{Z}\Tr[e^{-(\beta-\tau) H} d e^{-\tau H} d^\dagger]
\end{equation}
}

\frame { \frametitle{CT-QMC Expansion Algorithms}
There are several expansion algorithms with different choices of $H_b$. The most
widely used ones are:
\begin{itemize}
  \item CT-INT (Interaction expansion, $H_b=H^I_\text{loc}$): works well for clusters, single orbital models, 
  has sign problem with repulsive interactions, is not good for general electric structure Hamiltonians.
  \item CT-HYB (Hybridization expansion, $H_b=H_\text{hyb}$): works well for multi-orbital systems, 
  handles low temperature and strong interactions more efficiently, is not good for clusters.
\end{itemize}
Some other expansion algorithms that either consider an additional auxiliary field decomposition (for clusters) 
or exchange coupling (for Kondo problems) have been developed as well.
}