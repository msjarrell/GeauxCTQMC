\documentclass{article}

\begin{document}
\section{Exact diagonalization}
We need to evaluate the trace of this term:
\begin{equation}
  \label{eq:1}
  e^{-H*(\beta-t_n)}F_{t_n}e^{-H*(t_n-t_{n-1})}F_{t_{n-1}}\ldots F_{t_0}e^{-Ht_0}  
\end{equation}

We diagonalize the Hamiltonian with
\[
H=UVU^T
\]
where $V$ is diagonal matrix with eigenvalues of $H$, each column of $U$ is a
eigenvector of $H$.
Using
\[
UU^T=I
\]
we have 
\[
e^{-Ht}=e^{-UVU^Tt}=Ue^{-Vt}U^T
\]
 the term becomes
\[
Ue^{-V*(\beta-t_n)}U^TF_{t_n}Ue^{-V*(t_n-t_{n-1})}\ldots F_{t_0}Ue^{-Vt_0}U^T
\]
define 
\[
D_t=U^TF_tU
\]
The term is then
\begin{equation}
  \label{eq:2}
  Ue^{-V*(\beta-t_n)}D_{t_n}e^{-V*(t_n-t_{n-1})}\ldots D_{t_0}e^{-Vt_0}U^T  
\end{equation}

We can then evaluate the full trace of the matrix above.

\section{Krylov method}
The complexity of the above method is $O(m^3n)$, where $m$ is the size of the 
matrix, and $n$ is the number of fermion operators in the series. Since $m$ 
scales exponentially with the number of orbitals, this can be 
very expensive even for a moderate number of orbitals (say 5). Instead, we can 
use the Krylov method to find the trace.

First, we find the few lowest eigenstates of the Hamiltonian $|i\rangle$, since 
they are usually more relevant at low temperatures. Then the trace is 
approximately
\[
\sum_i\langle i|  e^{-H*(\beta-t_n)}F_{t_n}e^{-H*(t_n-t_{n-1})}F_{t_{n-1}}
\ldots F_{t_0}e^{-Ht_0}  |i\rangle
\]

Then each of the term in the summation become of a series of the following 
operations:
\begin{itemize}
\item $e^{-Ht}|v\rangle$
\item $F|v\rangle$
\end{itemize}

The second operation is $O(m^2)$, so we'll ignore it for now. For the first term,
we can generate a Krylov space using the following method: 
\footnote{ANALYSIS OF SOME KRYLOV SUBSPACE APPROXIMATIONS TO THE MATRIX 
EXPONENTIAL OPERATOR, Y. SAAD , section 2.1}
\begin{enumerate}
\item $v_1=v/||v||$,
\item Iteration: do $j=1,2,\ldots,k$ 
  \begin{enumerate}
  \item $w=Hv_j$
  \item Iteration: do $i=1,2...,j$
    \begin{enumerate}
    \item $h_{i,j}=w\cdot v_i$
    \item $w=w-h_{i,j}v_i$
    \end{enumerate}
  \item  $h_{j+1,j}=||w||$, $v_{j+1}=w/h_{j+1,j}$
  \end{enumerate}

\end{enumerate}

With these iteration, we generate a orthonormal basis 
$V_k=[v_1,v_2,\ldots, v_k]$
and a $k\times k$ matrix $H_k$, where $H_k(i,j)=h_{i,j}$.

The exponential term can be just evaluated by:
\[
e^{-Ht}v \approx ||v|| V_m e^{-H_k t}e_1
\]
where $e_1 = [1, 0, 0, \ldots 0]^T$ . 

The complexity of this operation is $O(k^3+mk^2+m^2k)$. Usually a small value ($\sim 3$)
of $k$ is needed, thus the complexity of the computation is reduced. 
Overall the complexity scales as $O(m^2kn)$.

\end{document}

%%% Local Variables:
%%% mode: latex
%%% TeX-master: t
%%% End:
